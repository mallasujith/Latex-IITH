\documentclass[12pt,-letter paper]{article}      
\usepackage{siunitx}                                              
\usepackage{setspace}
\usepackage{gensymb}                                              
\usepackage{xcolor}                                               
\usepackage{caption}
%\usepackage{subcaption}
\doublespacing                                                    
\singlespacing                              
\usepackage[none]{hyphenat}
\usepackage{amssymb}
\usepackage{relsize}
\usepackage[cmex10]{amsmath}
\usepackage{mathtools}
\usepackage{amsmath}                                              
\usepackage{commath}                                              
\usepackage{amsthm}
\interdisplaylinepenalty=2500
%\savesymbol{iint}
\usepackage{txfonts}                                              
%\restoresymbol{TXF}{iint}                                        
\usepackage{wasysym}                                              
\usepackage{amsthm}
\usepackage{mathrsfs}                                             
\usepackage{txfonts}                                              
\let\vec\mathbf{}
\usepackage{stfloats}
\usepackage{float}
\usepackage{cite}
\usepackage{cases}                                                
\usepackage{subfig}                                               
%\usepackage{xtab}
\usepackage{longtable}
\usepackage{multirow}
%\usepackage{algorithm}
\usepackage{amssymb}
%\usepackage{algpseudocode}
\usepackage{enumitem}
\usepackage{mathtools}
%\usepackage{eenrc}
%\usepackage[framemethod=tikz]{mdframed}                          
\usepackage{listings}                                             
%\usepackage{listings}
\usepackage[latin1]{inputenc}
%%\usepackage{color}{
%%\usepackage{lscape}
\usepackage{textcomp}
\usepackage{titling}
\usepackage{hyperref}
%\usepackage{fulbigskip}
\usepackage{tikz}
\usepackage{graphicx}                                             
\lstset{
  frame=single,
  breaklines=true
}
\let\vec\mathbf{}
\usepackage{enumitem}                                             
\usepackage{graphicx}                                             
\usepackage{siunitx}
\let\vec\mathbf{}                                                 
\usepackage{enumitem}
\usepackage{graphicx}
\usepackage{enumitem}
\usepackage{tfrupee}
\usepackage{amsmath}
\usepackage{amssymb}
\usepackage{mwe} % for blindtext and example-image-a in example
\usepackage{wrapfig}
\begin{document}
\title{PROBABILITY}
\author{Sujith 02}
\date{December 2023}
\maketitle
  \begin{enumerate}
	  \item If the probability of an event $E$ happening is $0.023$,then $P(\overline{E})=\underline{\hspace{2cm}}$.


	  \item Read the following passage and answer the questions given at the end:\newline\textbf{Diwali Fair}\newline A game in booth Diwali Fair involves using a spinner first.Then,if the spinner stops on an even member,the player is allowed to pick a marble from a bag.The spinner and the marbles in the bag are represented in \figref{fig:PROB.jpg}.\newline Prizes are given,when a block marble is picked.Shooter plays the game once.
  
		\begin{figure}[H]
			\centering
			\includegraphics[width=\columnwidth]{figs/PROB.jpg}
			\caption{A bag contains of marbles}
			\label{fig:PROB.jpg}
  \end{figure}
\begin{enumerate}[label=(\roman*)]
 \item what is the probability that she will be allowed to pick a mobile from the bag$?$
  \item suppose she is allowed to pick a marble from the bag,what if the probability of getting a prize,when it is given that the bag contains $20$ balls out of which $6$ came black$?$
\end{enumerate}
 \item A fair dice is thrown two times.Find the 
probability distribution of the number of sixes. 
Also determine the mean of the number of sixes.
 \item A card from a pack of $52$ cards is lost.From the remaining cards of the pack ,two cards are drawn randomly one-by-one without replacement and are found to be both kings.Find the probability of the last card being a king.
\end{enumerate}
\end{document}
